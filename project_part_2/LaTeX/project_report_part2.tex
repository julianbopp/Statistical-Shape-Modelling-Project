\documentclass[10pt]{article}
\usepackage[a4paper,bottom=3cm]{geometry}
\usepackage[english]{babel}
\usepackage[utf8]{inputenc}
\usepackage{amsmath}
\usepackage{amsthm}
\usepackage{amssymb}
\usepackage{graphicx}
\usepackage{subfig}
\usepackage{hyperref}

\theoremstyle{definition}
\newtheorem{definition}{Definition}[section]

\author{Julian Bopp, Takudzwa Togarepi}
\title{Sex and stature prediction from partial femurs using statistical shape modelling}
\begin{document}

\maketitle

\begin{abstract}
\noindent
Due to the femur bone's sheer size and strength, it holds immense significance within the human anatomy especially within the field of forensic science. Upon careful analysis, the femur bone offers invaluable insights into the identification of an individual's gender, stature and even geographic origin. This is due to the fact that femurs of individuals who are of the same geographic origin, gender and similar stature have similar femur bones. However, it is not unusual to come across incomplete or distorted femurs during the examination of human remains. This could be as result of death by accidents or other factors. This leads to a rise in the need for femur reconstruction, which in turn allows the modelling of complete femur structure and facilitates in-depth analysis. In an attempt to address this need, our project aims to  provide a model to effectively reconstruct partial femurs and use them to predict the gender and the stature of whoever the femurs belonged to.\\

\noindent
$\bold{Keywords}$: femur, forensics, identification, gender prediction, reconstruction, modeling, stature.\\

\end{abstract}
\section{Introduction}

We use Bayesian model fitting techniques to reconstruct a complete bone from a partial femur. This means that we find the parameters of the shape model that best describe the partial bone. Given this reconstruction we can infer the stature and sex of the person that was the original host of the bone.
\section{Methods}



\section{Experiments and results}

\subsection{Data and experimental setup}

Explain the data with which you are experimenting.
Highlight aspects of the data that are particularly interesting 
for the project. 

\subsection{Experimental results}


\section{Conclusion}

Add your conclusion here. What is the main result? What did you achieve, what 
needs to be done. 


\end{document}
