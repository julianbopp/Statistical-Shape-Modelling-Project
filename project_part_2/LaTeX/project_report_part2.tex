\documentclass[10pt]{article}
\usepackage[a4paper,bottom=3cm]{geometry}
\usepackage[english]{babel}
\usepackage[utf8]{inputenc}
\usepackage{amsmath}
\usepackage{amsthm}
\usepackage{amssymb}
\usepackage{graphicx}
\usepackage{subfig}
\usepackage{hyperref}

\theoremstyle{definition}
\newtheorem{definition}{Definition}[section]

\author{Julian Bopp, Takudzwa Togarepi}
\title{Sex and stature prediction from partial femurs using statistical shape modelling}
\begin{document}

\maketitle

\begin{abstract}
\noindent
Due to the femur bone's sheer size and strength, it holds immense significance within the human anatomy especially within the field of forensic science. Upon careful analysis, the femur bone offers invaluable insights into the identification of an individual's gender, stature and even geographic origin. This is due to the fact that femurs of individuals who are of the same geographic origin, gender and similar stature have similar femur bones. However, it is not unusual to come across incomplete or distorted femurs during the examination of human remains. This could be as result of death by accidents or other factors. This leads to a rise in the need for femur reconstruction, which in turn allows the modelling of complete femur structure and facilitates in-depth analysis. In an attempt to address this need, our project aims to  provide a model to effectively reconstruct partial femurs and use them to predict the gender and the stature of whoever the femurs belonged to.\\

\noindent
$\bold{Keywords}$: femur, forensics, identification, gender prediction, reconstruction, modeling, stature.\\

\end{abstract}
\section{Introduction} From a set of given femur data we train a regression model to predict sex and stature.
We then use Bayesian model fitting techniques to be able reconstruct a complete bone from a partial femur. This means that we find the parameters of the shape model that best describe the partial bone. Given this reconstruction we can aquire the necessary measurements of the bone to use our regression model to predict the sex and stature from a partial bone.

\section{Methods}
We introduce the problem setting and list the methods used to solve it.

\begin{definition}[Bayesian linear regression]
We assume that a variable $y$ is modelled as a linear function of $x$. More precisely our assumption is
\begin{equation}
y \sim N(a\cdot x + b, \sigma^2).
\end{equation}
Let $\theta = (a,b,\sigma^2)$. Given $X=(x_1,\dots,x_n)$ and $Y=(y_1,\dots,y_n)$, we want to estimate $\theta$ using the posterior
$$p(\theta|X,Y) \propto p(Y|\theta,X)p(\theta).$$
\end{definition}

The way we solve the Bayesian linear regression model in scalismo is by specifying priors over the parameters and using the Metropolis Hastings algorithm. The Metropolis Hastings algorithm allows us to iteratively draw samples from the posterior distribution.

\begin{definition}[Metropolis Hastings Algorithm]
Given an initial sample $x$ and a proposal distribution $Q$ that is used to propose new samples we do the following steps:
\begin{enumerate}
\item Draw a sample $x'$ from $Q(x'|x)$.
\item With probability $$\alpha=\text{min}\{\frac{p(x')Q(x|x')}{p(x)Q(x'|x)},1\}$$
accept $x'$ as new state $x$.
\item Emit current state $x$ as sample
\end{enumerate}


\end{definition}







\section{Experiments and results}

\subsection{Data and experimental setup}
In this experiment, two datasets are used. One consists of a table of 42 measurements of stature, femur-bone-length, trochanter-distance and sex of a person. The second data set consists of 10 3D partial femurs meshes. We first train a linear and logistic regression model using the metropolis hastings algorithm, on the the measurements from the table.
\subsection{Experimental results}


\section{Conclusion}

Add your conclusion here. What is the main result? What did you achieve, what 
needs to be done. 


\end{document}
