\documentclass[10pt]{article}
\usepackage[a4paper,bottom=3cm]{geometry}
\usepackage[english]{babel}
\usepackage[utf8]{inputenc}
\usepackage{amsmath}
\usepackage{amssymb}
\usepackage{graphicx}

\author{Takudzwa Togarepi, Julian Bopp }
\title{Project Part 1: Probabilistic modeling of the femur anatomy}
\begin{document}

\maketitle
\begin{abstract}
    Describe the project in a few sentences. 
    The reader should get an idea of what the project is all about
    and what you achieved in the project.


    In this first part of the project ''Sex and stature prediction from partial femurs using statistical shape modelling'' we build,
    analyze and validate a model of the femur shape.
\end{abstract}

\section{Introduction}

We build, analyze and validate a model of the femur anatomy. Probabilistic modeling of femur anatomy has important applications in medical imaging, including computer-assisted surgery, patient-specific implant design, and bone fracture analysis. By accurate modeling of anatomical landmarks on the femur, these techniques can help improve surgical outcomes and reduce complications. In particular we construct a Gaussian Process (GP) Model using a dataset consisting of data from 46 femurs.
\section{Methods}

In this section you can describe the methods you used to solve the problem.


\section{Experiments and results}

\subsection{Data and experimental setup}

Explain the data with which you are experimenting.
Highlight aspects of the data that are particularly interesting 
for the project. 

\subsection{Experimental results}


\section{Conclusion}

Add your conclusion here. What is the main result? What did you achieve, what 
needs to be done. 


\end{document}
